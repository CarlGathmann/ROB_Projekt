\documentclass[conference]{IEEEtran}
\IEEEoverridecommandlockouts
\usepackage{biblatex}
\usepackage{amsmath,amssymb,amsfonts}
\usepackage{algorithmic}
\usepackage{graphicx}
\usepackage{textcomp}
\usepackage{xcolor}
\usepackage{enumitem}

\addbibresource{Bacherlor_Seminar.bib}

\begin{document}

\title{Haptisches Feedback eines Roboters durch virtuelle 3D-Modelle}

\author{
    \IEEEauthorblockN{Carl Gathmann}
    \IEEEauthorblockA{\textit{Universität zu Lübeck}\\
        Lübeck, Germany \\
        carl.gathmann@student.uni-luebeck.de}
    \and
    \IEEEauthorblockN{Marten Buchmann}
    \IEEEauthorblockA{\textit{Universität zu Lübeck}\\
        Lübeck, Germany \\
        marten.buchmann@student.uni-luebeck.de}
}
\maketitle

\begin{abstract}
    !!!NOCH GPT!!!
    In dieser Studie wird ein neuartiger Ansatz zur Generierung von haptischem Feedback in Mensch-Roboter-Interaktionen vorgestellt. Mittels maßgeschneiderter Software erzeugen wir eine sensorische Wahrnehmung, die durch frei gestaltbare, virtuelle 3D-Modelle gesteuert wird. Der Anwender erfährt eine Art abweisende Kraft, die durch die räumlichen Grenzen des virtuellen Modells definiert ist, wodurch eine physische Interaktion mit dem immateriellen Modell simuliert wird. Die vorgestellte Technologie findet breite Anwendungsbereiche, von medizinischen Simulationen und Exploration in gefährlichen Zonen bis hin zur Erhöhung der Spielerfahrung in virtuellen Umgebungen. In Verbindung mit Virtual-Reality-Ausrüstung eröffnet unsere Methode neue Wege zur Verbesserung der Benutzerimmersion durch die Vermittlung eines realistischeren Gefühls für die Form und Textur virtueller Objekte. Die in dieser Arbeit vorgestellten Prinzipien und Implementierungen können als Grundlage für weiterführende Forschungen und Entwicklungen auf diesem aufstrebenden Gebiet dienen.
\end{abstract}

\begin{IEEEkeywords}
    component, formatting, style, styling, insert
\end{IEEEkeywords}

\section{Introduction}
!!! GPT =>

Die Mensch-Roboter-Interaktion (MRI) hat in den letzten Jahren zunehmend an Bedeutung gewonnen und sich als ein dynamisches Forschungs- und Anwendungsfeld etabliert. Im Zentrum dieser Interaktion steht die Verbesserung der Benutzererfahrung durch die Erweiterung der sinnlichen Wahrnehmung des Menschen.

In dieser Studie präsentieren wir einen Ansatz, der es dem Benutzer ermöglicht, virtuelle 3D-Modelle zu "ertasten", indem er über ein Roboterinterface mit ihnen interagiert. Dies wird durch die Erzeugung einer abweisenden Kraft erreicht, die auf den Grenzen der virtuellen Modelle basiert. Dieses haptische Feedback simuliert das physische Berühren eines realen Objekts, obwohl kein tatsächlicher physischer Kontakt mit dem virtuellen Modell besteht. 

Dieser Ansatz bietet zahlreiche Anwendungsmöglichkeiten, darunter die Simulation von Operationen für Ausbildungszwecke, die Erkundung von Objekten in gefährlichen oder unzugänglichen Umgebungen und die Erhöhung der Immersion in virtuellen Spielen. Darüber hinaus kann die Kombination unserer Technologie mit Virtual-Reality-Brillen zu einem verbesserten Gefühl von Präsenz und Realismus in virtuellen Umgebungen führen.

Das vorliegende Paper beleuchtet die zugrundeliegenden Prinzipien, technische Details und potenzielle Anwendungen dieses Ansatzes. Es leistet einen wertvollen Beitrag zur Erforschung und Weiterentwicklung von Technologien für haptisches Feedback in der Mensch-Roboter-Interaktion.

<= GPT !!!

\section{Methods}

Die Erzeugung des haptischen Feedbacks beruht auf der Berechnung einer abweisenden Kraft, die so konzipiert ist, dass sie dem Benutzer ein realistisches Gefühl für die Form des virtuellen 3D-Modells vermittelt. Der Kern dieser Technik basiert auf der Nutzung des weit verbreiteten und flexiblen STL-Standards, der allgemein in der Industrie zur Speicherung und Übertragung von 3D-Modellen eingesetzt wird.

Für die physische Implementierung der Mensch-Roboter-Interaktion wurde der Panda-Roboter von Franka Emika ausgewählt. Der Panda zeichnet sich durch seine 7-achsige Struktur aus, die hohe Präzision und Flexibilität bietet. Eine benutzerfreundliche, 3D-gedruckte Schnittstelle (\ref{fig:nullSpace}) wurde am Endeffektor des Roboters installiert, um die Führung durch menschliche Benutzer zu erleichtern.  

\begin{figure}
    \centering
    \includegraphics[width=0.45\textwidth]{pics/interface.jpeg}
    \caption{Mensch-Roboter-Schnittstelle}
    \label{fig:nullSpace}
\end{figure}

Bei der Entwicklung der Software für dieses System stand die Leistungsfähigkeit im Vordergrund, um eine nahtlose Benutzererfahrung zu gewährleisten. Dazu war es entscheidend, dass die Anwendung in Echtzeit ausgeführt werden kann, wobei eine maximale Zeitverzögerung von weniger als 1 Millisekunde akzeptiert wurde. Die Programmiersprache C++ wurde aufgrund ihrer hohen Performance ausgewählt, um diesem Anspruch gerecht zu werden. Die Hardware-Konfiguration bestand aus [!!!HARDWARE!!!].

Die Verarbeitung und Nutzung der 3D-Modelle wurden durch die Umwandlung der STL-Daten in Eigen-Matrizen optimiert. Dieser Prozess wurde durch einen STL-Parser ermöglicht und ermöglichte eine einfache und effiziente Handhabung der 3D-Daten innerhalb des Systems.

Insgesamt zeichnet sich unsere Methodik durch die Kombination von Standardtechnologien und innovativen Ansätzen aus, um eine effiziente und benutzerfreundliche Lösung für haptisches Feedback in der Mensch-Roboter-Interaktion zu schaffen.

\section{Results}

\section{Discussion}

\section{Conclusion}

\printbibliography

\end{document}